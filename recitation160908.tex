\documentclass{article}

\usepackage{amsfonts}
\usepackage{amsmath}
\usepackage{bbm}

\usepackage{xcolor}
\newenvironment{solution}{\color{red}}{\color{black}}

\begin{document}

\title{Recitation 1}
\author{Michael Chirico}

\maketitle

\section{Return of the Pigs}

Recall Pass the Pigs from last week. Note the new column denoting point values for each roll.

\begin{table}[htbp]
\centering
\begin{tabular}{|r|c|c|c|}
\hline
Roll Type & Number of Rolls & Proportion & Point Value \\
\hline
Blank & 1,344 & .341 & 0 \\
\hline
Dot & 1,294 & .329 & 1 \\
\hline
Razorback & 767 & .195 & 5 \\
\hline
Trotter & 365 & .092 & 5 \\
\hline
Snouter & 137 & .035 & 5 \\
\hline
Leaning Jowler & 32 & .008 & 10 \\
\hline
\end{tabular}
\caption{Empirical Likelihood of Pig Roll Types}
\label{tbl:pigout}
\end{table}

What is the expected value of a roll in this game? Variance?

\begin{solution}
The expected value is:

\[ .341(0) + .329(1) + .195(5) + .092(5) + .035(5) + .008(10) \approx  2\]

We'll use the quick formula for the variance, $\mathbb{V}[X] = \mathbb{E}[X^2] - \left( \mathbb{E}[X] \right)^2$, for which we need to calculate $\mathbb{E}[X^2]$:

\[ .341(0^2) + .329(1^2) + .195(5^2) + .092(5^2) + .035(5^2) + .008(10^2) \approx 9 \]

So $\mathbb{V}[X] \approx 9 - 4 \approx 5$.
\end{solution}

\section{Backgammon}

In backgammon, your turn starts with a roll of two dice. You are able to move your chips as many spaces as the sum of the two faces, with one backgammon-specific twist. Doubles count double. So, for example, rolling 4-4 means you can move $4+4+4+4=16$ total spaces, but a roll of 1-6 means that you can move $6+1=7$ spaces.

\begin{enumerate}
\item What is the expected number of spaces you can move on a single turn in Backgammon?
\item A backgammon board is 24 spaces long. How many turns do you expect it to take to make a full run for two chips?
\end{enumerate}

\begin{solution}
\begin{enumerate}
\item We know the average of a standard dice roll is 7, which is from $2 + 3 + 3 + \ldots + 11 + 11 + 12$ (adding all possible outcomes) divided by 36 (the number of possible outcomes), so the sum must be $7\cdot 36 = 252$. To this total, we must add the total pips that are bonused as a result of rolling doubles: $2 + 4 + 6 + 8 + 10 + 12 = 42$ (when we roll 1-1, we get $1+1+1+1$, but the first two of these were taken care of in arriving to 252; the latter two were as yet uncounted, and are now covered by the 2 in $2+\ldots+12$).

Thus the numerator is now $252 + 42 = 294$, so the average is $\frac{294}{36}\approx 8.17$ -- slightly higher than 7, as we might have anticipated.

\item Two full chips require 48 total spaces, or $approx \frac{48}{\frac{294}{36}} \approx 6$.
\end{enumerate}
\end{solution}

\section{Linear Combinations}

Consider the discrete RVs $X$ and $Y$ with PMFs:

\begin{table}[htbp]
\centering
\begin{tabular}{r|ccc}
$x$ & $-1$ & $0$ & $1$ \\
\hline
$\mathbb{P}[X = x]$ & $.05$ & $.4$ & $.55$ \\
\end{tabular}
\caption{PMF of $X$}
\label{tbl:x_dist}
\end{table}

\begin{table}[htbp]
\centering
\begin{tabular}{r|ccccc}
$y$ & $2$ & $3$ & $5$ & $7$ & $11$ \\
\hline
$\mathbb{P}[Y = y]$ & $.1$ & $.1$ & $.1$ & $.1$ & $.6$ \\
\end{tabular}
\caption{PMF of $X$}
\label{tbl:y_dist}
\end{table}

\begin{enumerate}
\item Describe the distributions of $X$ and $Y$ at a glance.
\item Draw the CDFs of $X$ and $Y$.
\item What is $\mathbb{E}[X]$?
\item What is $\mathbb{E}[X^2]$?
\item What is $\mathbb{V}[Y]$?
\item What is $\mathbb{E}[X^2 - \frac12 Y]$?
\item What is $\mathbb{E}[Z \mid Z > 5]$, where $Z = X + |Y - 5|$
\end{enumerate}

\begin{solution}
\begin{enumerate}
\item $X$ is highly concentrated on 1; $Y$ is highly concentrated on 11, but uniform on other values.
\item Step functions
\item $.05(-1) + .4(0) + .55(1) = .5$
\item $.05(1) + .55(1) = .6$
\item $\mathbb{E}[Y] = .1(2 + 3 + 5 + 7) + .6(11) = 8.3$; $\mathbb{E}[Y^2] = .1(4 + 9 + 25 + 49) + .6(121) = 81.3$, so $\mathbb{V}[Y] = 81.3 - 8.3^2 = 12.41$
\item $\mathbb{E}[X^2 - \frac12 Y] = \mathbb{E}[X^2] - \frac12 \mathbb{E}[Y] = .6 - \frac12 8.3 = -3.55$
\item Technically, we need to know that $X$ and $Y$ are independent, which I forgot to mention. $X + |Y - 5|$ can only more than 5 if $Y = 11$ and ($X = 0$ or $X = 1$). So if $Z$ is more than 5, its expected value is: $\frac{.4}{.4 + .55}(6) + \frac{.55}{.4 + .55}(7) \approx 6.6$.
\end{enumerate}
\end{solution}

\section{Continuous Distribution}

Suppose that the random variable $X$ has pdf

\[ p(x) = \frac12 (x - 1) \mathbbm{1}[1 < x < 3] \]

(the $\mathbbm{1}[1 < x < 3]$ part is just a convenient way of expressing that the domain of this RV is $[1, 3]$ in-line instead of needing to define it piecewise)

\begin{enumerate}
\item Find a monotone function $u(x)$ such that the random variable $Y = u(x)$ is distributed $U[0,1]$.
\item Think of how you'd write a computer program to generate draws from the random variable $X$. \textit{Hint: you can start from $U[0, 1]$ draws and then convert them into draws of $X$.} 
\end{enumerate}

\begin{solution}
$u = F^{-1}(x)$, where $F^{-1}$ is the inverse of the CDF of $X$. 
\end{solution}

\end{document}