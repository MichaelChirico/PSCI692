\documentclass{article}

\usepackage{amsfonts}

\begin{document}

\title{Recitation 1}
\author{Michael Chirico}

\maketitle

\section{Pass the Pigs}

Pass the pigs is a silly dice-like game now produced by Xingcolo, once produced by Milton Bradley:

https://www.amazon.com/exec/obidos/ASIN/B00005JG3Y/probabilitandpig

Players take turns rolling a pair of dice-sized rubber pigs. Depending on how they land, you accumulate points. On each turn, you continue to roll and accumulate points until you choose to pass or until a ``stop'' formation is rolled, at which point your turn ends automatically and you forfeit all points gained on that turn. 

Like a dice, there are six possible outcomes of a roll of a single pig; unlike a dice, the probability of each occurring is not equal.

It's not clear \textit{ex ante} what the exact probability of each outcome actually is; luckily some time-rich fellow named Freddie W. paid some students to conduct an experiment consisting of 3,939 rolls of the pigs, which gives a sort of empirical distribution of the outcomes, show in Table \ref{tbl:pigout}:

\begin{table}[htbp]
\centering
\begin{tabular}{|r|c|c|}
\hline
Roll Type & Number of Rolls & Proportion \\
\hline
Blank & 1,344 & .341 \\
\hline
Dot & 1,294 & .329 \\
\hline
Razorback & 767 & .195 \\
\hline
Trotter & 365 & .092 \\
\hline
Snouter & 137 & .035 \\
\hline
Leaning Jowler & 32 & .008 \\
\hline
\end{tabular}
\caption{Empirical Likelihood of Pig Roll Types}
\label{tbl:pigout}
\end{table}

\begin{enumerate}
\item What is the probability of rolling a single razorback? (Recall that a roll consists of rolling two pigs at once)
\item What is the probability of rolling a pig-out (one Blank and one Dot)?
\item What is the probability of rolling a double leaning jowler?
\item What is the probability of rolling mixed (neither pig is a Dot or a Blank, and the pigs are different)?
\end{enumerate}

\section{Refugee Screening}

There are about 7 billion people in the world. About 23 million of them live in Syria.

Let's say there are 200,000 terrorists in the world, 30,000 of whom are in Syria.

\begin{enumerate}
\item Given these rough estimates, what is the probability that a randomly selected person in the world is a terrorist?
\item Given that someone is a terrorist, what is the probability that they are Syrian?
\item You're screening refugees and hoping to weed out any terrorist posing as a refugee. Given that you know the refugee is Syrian, what is the probability they are a terrorist? That is, what is $\mathbb{P}[terrorist|Syrian]$?
\item \textit{(open-ended/framing thoughts moving forward)} This is of course a simplistic interpretation of the problem of refugee screening. Why does the simplistic analysis break down? What are other important considerations? What assumptions are violated?
\end{enumerate}

\end{document}