\documentclass{article}

\usepackage{amsmath}
\DeclareMathOperator*{\Max}{Max}
\newcommand{\st}{\hbox{ s.t. }}

\usepackage{graphicx}

\begin{document}

\title{Recitation 5}
\author{Michael Chirico}

\maketitle

\section*{Welfare Maximization}

Suppose (throwing an entire course in graduate-level economics into a black box) that a society's preferences over schools, $S$, and parks, $P$, are given by the following \textit{welfare function}:

\[ W(S, P) = \frac34 \ln S + \frac14 \ln P \]

Suppose further that the government has \$100 million to allocate between these two uses of public funds.

Each school costs \$25 million, and each park costs \$5 million.

What is the welfare-maximizing allocation of schools and parks?

The formal problem that we're solving is

\[ \Max_{S, P} \left\{ W(S, P) \right\} \qquad \st 25S + 5P \leq 100; \qquad S, P \geq 0 \]

We'll focus on the final part of solving this problem. First notice that, since welfare is increasing in both $S$ and $P$, the government will spend all \$100 million (in more common terms -- society is improved by any additional school or park, so there is no incentive not to spend all of the money).

In math, this means that

\[ 25S + 5P = 100 \]

\textit{(Be sure you understand what this equation means!)} 

With this, we can eliminate one of the two choice variables, say $P$, by replacing $P$ everywhere with $P = 20 - 5S$. Thus the problem simplifies to:

\[ \Max_S \left\{ \frac34 \ln S + \frac14 \ln \left( 20 - 5S \right) \right\} \]

We're trying to maximize the function $W(S) \equiv \frac34 \ln S + \frac14 \ln \left( 20 - 5S \right)$ with respect to $S$. What on earth does this function look like?

Can plot with R or Wolfram Alpha, with output in Figure \ref{fig:welfare}:

\begin{figure}[htbp]
\centering
\includegraphics[width = .5\linewidth]{recitation160929_fig1.png}
\caption{Appearance of Welfare Function at Budget Line}
\label{fig:welfare}
\end{figure}

Our goal is to identify analytically the exact place where this function is maximized -- visually, it's in the neighborhood of $S = 3$.

\begin{enumerate}
\item What is the analytic/calculus condition that describes the value of $S$ that achieves the maximum of this function?
\item What is the derivative of $W$ with respect to $S$?
\item What value of $S$ eliminates the derivative (i.e., for which $S$ is $W'(S) = 0$)?
\item \textit{(Complete in your free time)} What is the second derivative of $W$ with respect to $S$, and what can we learn from it about the critical point discovered in part 3)? 
\end{enumerate}

\section*{Some Matrix Algebra}

Express the following system of equations as a single matrix equation of the form $AX = B$:

\begin{alignat*}{4}
2x & {}+{} &  y & {}+{} & 2z & {}={} & 10 \\
 x & {}+{} &  y & {}+{} &  z & {}={} &  6 \\
 x & {}+{} & 3y & {}+{} & 2z & {}={} & 13
\end{alignat*}

The solution to the matrix equation $AX = B$ is given by $X = A^{-1}B$.

\begin{enumerate}
\item Find $A^{-1}$.
\item Find $A^{-1}B$.
\end{enumerate}

\end{document} 